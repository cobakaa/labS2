\documentclass[a4paper]{article}
\usepackage[left=25mm, right=10mm, bottom=10mm, nohead, nofoot]{geometry}

\usepackage[utf8]{inputenc}
\usepackage[english,russian]{babel}
\usepackage{calc}
\usepackage{setspace}
\usepackage{fixltx2e}
\usepackage{graphicx}
\usepackage{wrapfig}
\usepackage[normalem]{ulem}
\usepackage{color}
\usepackage{hyperref}
\usepackage{tikz} 
\usetikzlibrary{calc,arrows.meta}
\usepackage[strict]{changepage}
\usepackage[14pt]{extsizes}
\usepackage[absolute]{textpos}
\usepackage{libertine}


\linespread{0.9}
\topmargin=-1cm
\oddsidemargin=30pt
\evensidemargin=30pt
\textheight=23cm
\textwidth=13.8cm

\usepackage{sectsty}

\sectionfont{\fontsize{15}{15}\selectfont}



\newcommand{\RomanNumeralCaps}[1]
{\MakeUppercase{\romannumeral #1}}

\makeatletter
\renewcommand{\@oddfoot}{\hfil
	\begin{tikzpicture}
		\draw[thick] (0,0) -- (2.4,0)  +(-1.3,-0.75) node[above] {\textit{ \arabic{page}}};
	\end{tikzpicture}
\hfil}
\renewcommand{\@evenfoot}{\hfil 	
	\begin{tikzpicture}
		\draw[thick] (0,0) -- (2.4,0)  +(-1.3,-0.75) node[above] {\textit{ \arabic{page}}};
	\end{tikzpicture}
\hfil}
\makeatother

\setcounter{page}{39}


\renewcommand{\rmdefault}{LinuxLibertineT-TLF}

\begin{document}

%\fontfamily{DejaVuSerif-TLF}

\begin{wrapfigure}[5]{r}{0.3\linewidth} 
	
	\begin{tikzpicture}[scale=1.4]
		\draw[very thick, -{Latex[length=6mm, width=2mm]}] (-0.4,0) -- (2.5,0);
		
		\foreach \x in {0,1}
		{        
			\coordinate (A\x) at ($(0,0)+(\x*1.4cm,0)$) {};
			\draw ($(A\x)+(0,2pt)$) -- ($(A\x)-(0,2pt)$);
			\node at ($(A\x)-(0,2ex)$) {\x};
		}
		
		\fill[black] (0.4,0) circle (1.5pt) -- +(0,0) node[above] {$a$};
		\fill[black] (1.8,0) circle (1.5pt) -- +(0,0) node[above] {$b$};
		
		
	\end{tikzpicture}
\vspace{0.3cm}
\hspace{1cm}
	Рис. 3
\end{wrapfigure}

Для вдумчивого читателя заметим, что ссылка в начале параграфа на то, что действительные числа и их свойства известны из курса математики, не является необходимой. Сформулированные выше свойства действительных чисел можно взять за исходное определение. Следует только исключить тривиальный случай: легко проверить, что для множества, состоящего только из одного нуля, выполняются все свойства \RomanNumeralCaps{1}--\RomanNumeralCaps{5} (в таком множестве $0 = 1$). Множество, в котором имеется хоть один элемент, отличный от нуля, будем здесь для кратности называть нетривиальным.

Перефразируя сказанное, получим следующее определение.

\noindent\textbf{Определение 1}. \textit{Нетривиальное множество элементов, обладающих свойствами \RomanNumeralCaps{1}--\RomanNumeralCaps{5}, называется множеством действительных чисел. Каждый элемент этого множества называется действительным числом.}

Построение теории действительных чисел, основывающееся на таком их определении, называется \textit{аксиоматическим}, а свойства \RomanNumeralCaps{1}--\RomanNumeralCaps{5} -- \textit{аксиомами действительных чисел.}

Геометрически множество действительных чисел изображается направленной (ориентированной) прямой, а отдельные числа -- ее точками (рис. 3).

При такой интерпретации действительных чисел иногда вместо $a$ меньше $b$ ($a$ больше $b$) говорят, что точка $a$ лежит левее точки $b$ (что $a$ лежит правее $b$).

В п. $2.2^*$ -- $2.6^*$ будут более детально проанализированы свойства \RomanNumeralCaps{1}--\RomanNumeralCaps{5} действительных чисел и выведены некоторые их следствия. Как и все пункты, отмеченные звездочками, эти пункты при первом чтении можно без существенного ущерба для усвоения курса математического анализа опустить. Для понимания дальнейшего материала (в \S 3 и следующих) вполне достаточно представления о действительных числах, которое дается в курсе элементарной математики.

\section*{\boldmath$2.2^*$. Свойства сложения и умножения}

Рассмотрим некоторые свойства сложения и умножения, которые вытекают из свойств \RomanNumeralCaps{1}, \RomanNumeralCaps{2} и \RomanNumeralCaps{3}. Прежде всего заметим,


\end{document}